\documentclass[]{article}

%opening
\title{Grundlagen der Startplanung}
\author{Ida Hönigmann}

\newenvironment{question}{\vspace{3mm}\noindent\bfseries}{\\}

\begin{document}

\maketitle

\section{Einführung in das Semesterprogramm}

\section{Wien - Geschichte, Gegenwart, Zukunftsperspektiven}
\begin{question}
	Benennen Sie einige zentrale Phasen der Wiener Stadtentwicklung und richten Sie den Fokus dabei vor allem auf die Entwicklung ab Mitte des 19. Jahrhunderts!
\end{question}
Wien entwickelte sich auf dem Ort eines römischen Legionslagers zu einer immer bevölkerungsreicheren Stadt. Aus Verteidigungszwecken wurde ein Befestigungswall um die Stadt errichtet, was jedoch ab dem 19. Jahrhundert zu einem Platzmangel führt. Eines der entscheidenden Projekte für die Stadt Wien ist daher der Umbau der Verteidigungsmauern zur Ringstraße.

Ein weiteres großes Projekt war die Regulierung der Donau. Vor und im 19. Jahrhundert kam es immer wieder zu Überschwemmungen, weswegen immer mehr Arme der Donau abgesperrt wurden. Nach dem Hochwasser im Jahr 1954 war klar, dass zusätzlich ein Entlastungsgerinne notwendig war. Der Bau dieses Neuen Donau ließ die sogenannte Donauinsel entstehen.

Im Zuge von Stadterweiterungen wird Anfang des 20. Jahrhunderts ist immer wieder vom kreisförmigen Stadtbild und dem grünen Ring um Wien die rede.

Viele Wiener lebten um 1900 in schlechten Wohnbedingungen. Um die Wohnungsnot zu bekämpfen werden in der Zeit des roten Wiens Wohnungsbauten von der Gemeinde Wien gebaut. Diese Wohnungen entsprechen wesentlich besseren Standards als davor in Wien üblich. Es wird begonnen Richtlinien und Bestimmungen in Form von Bauordnungen zu verordnen.

Nach dem zweiten Weltkrieg entstanden durch den steigenden Mobilisierungsgrad ermöglicht viele Entwicklungen an den Rändern der Stadt Wien. Es wurde Auto-freundlich gebaut und umgebaut. Allerdings wurde diese Entwicklung auch kritisch gesehen. So wird der Fokus der Stadtentwicklung in Wien auf die Erhaltung und Erneuerung gelegt.

Nach dem Fall der eisernen Mauer rückt Wien weiter ins Zentrum Europas. Es kommt immer wieder Überlegungen über eine Kooperation mit Bratislava.

\begin{question}
	Worin begründen sich die besonderen Herausforderungen der Wiener Stadtentwicklung?
\end{question}
Historisch wurde die räumliche Begrenzung aufgrund der Befestigungsanlage und die immer wiederkehrenden Überflutungen der Donau herausfordernd gewesen. Auch der Wohnungsmangel, eine Schwankende Bevölkerungsgröße (EU-Erweiterung, Suburbanisierung), Änderungen der Anforderungen an die bauliche Struktur und der Altersverteilung der Bevölkerung stellen Herausforderungen dar.

Entstehen neuer Zentren, Steigende Bedeutung öffentlicher Raum und Grätzel, Klimawandel, Demografische, soziale und ökonomische Veränderung, Veränderung in der Mobilität und Mobilitätsverhalten, Klimaschutz, Wandel des Einzelhandel

\begin{question}
	Erläutern Sie wesentliche im STEP 2025 dokumentierte Prinzipien der Wiener Stadtentwicklung!
\end{question}
Neue Entwicklungen sollen allen zu gute kommen

\begin{question}
	Erläutern Sie die wesentlichen Ziele der vier Handlungsbereiche des STEP 2025!
\end{question}
Wir leisten uns Stadt: lebenswert, sozial gerecht, geschlechtergerecht, bildung, weltoffen, prosperierende, integrierte, ökologische, partizipative Stadt(region)

Wien baut auf

Wien wächst über sich hinaus

Wien ist vernetzt

\section{Theorie und Methodik der Stadtplanung}

\section{Instrumente der Örtlichen Raumplanung}

\end{document}
