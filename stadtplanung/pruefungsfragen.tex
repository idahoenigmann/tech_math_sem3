\documentclass[]{article}
\usepackage[margin=2.5cm]{geometry}

%opening
\title{Grundlagen der Startplanung}
\author{Ida Hönigmann}

\newenvironment{question}{\vspace{8mm}\noindent\bfseries}{\\}

\begin{document}

\maketitle

\section{Einführung in das Semesterprogramm}
\begin{question}
	Erläutern Sie wesentliche Faktoren, die die Entwicklung der Stadt beeinflussen. Gruppieren Sie diese.
\end{question}
TODO

\begin{question}
	''Stadtplanung ist eine Wissenschaft, eine Kunst, eine politische Bestrebung.'' Diskutieren Sie dies.
\end{question}
Stadtplanung ist eine Wissenschaft, eine Kunst und eine politische Bestrebung, die sich auf die Formung und Lenkung des physischen Wachstums und der Ordnung von Städten in Einklang mit ihren sozialen und wirtschaftlichen Bedürfnissen richtet.

Wir betreiben sie
\begin{itemize}
	\item als Wissenschaft, Kenntnisse der Stadtstruktur, ihrer Dienstleistungen sowie der Beziehung ihrer Bestandteile und der Verkehrsbewegungen zu gewinnen;
	\item als Kunst mit dem Ziel der Bestimmung der Bodenordnung, der Anordnung von Flächennutzungen und Verkehrswegen und des Gebäudeentwurfes nach Grundsätzen, die Ordnung, Gesundheit und Wirtschaftlichkeit sichern;
	\item und als politische Bestrebung, um unseren Grundsätzen Wirksamkeit zu verliehen.
\end{itemize}

\begin{question}
	Was verstehen Sie unter ''Ordnungsaufgaben'', was unter ''Gestaltungsaufgaben'' in der Stadtplanung?
\end{question}
\begin{itemize}
	\item Ordnungsaufgaben: Fragen des Flächenanspruchs und der wechselseitigen Zuordnung verschiedener Nutzungen
	\item Gestaltungsaufgaben: dreidimensionale Gestaltung des städtischen Raumes
\end{itemize}

\section{Wien - Geschichte, Gegenwart, Zukunftsperspektiven}
\begin{question}
	Erläutern Sie die zentralen Rahmen- und Ausgangsbedingungen zum STEP 2025 in Wien. Wie reagiert der STEP auf diese Herausforderungen?
\end{question}
TODO

\begin{question}
	Benennen Sie einige zentrale Phasen der Wiener Stadtentwicklung und richten Sie den Fokus dabei vor allem auf die Entwicklung ab Mitte des 19. Jahrhunderts!
\end{question}
Wien entwickelte sich auf dem Ort eines römischen Legionslagers zu einer immer bevölkerungsreicheren Stadt. Aus Verteidigungszwecken wurde ein Befestigungswall um die Stadt errichtet, was jedoch ab dem 19. Jahrhundert zu einem Platzmangel führt. Eines der entscheidenden Projekte für die Stadt Wien ist daher der Umbau der Verteidigungsmauern zur Ringstraße.

Ein weiteres großes Projekt war die Regulierung der Donau. Vor und im 19. Jahrhundert kam es immer wieder zu Überschwemmungen, weswegen immer mehr Arme der Donau abgesperrt wurden. Nach dem Hochwasser im Jahr 1954 war klar, dass zusätzlich ein Entlastungsgerinne notwendig war. Der Bau dieses Neuen Donau ließ die sogenannte Donauinsel entstehen.

Im Zuge von Stadterweiterungen wird Anfang des 20. Jahrhunderts ist immer wieder vom kreisförmigen Stadtbild und dem grünen Ring um Wien die rede.

Viele Wiener lebten um 1900 in schlechten Wohnbedingungen. Um die Wohnungsnot zu bekämpfen werden in der Zeit des roten Wiens Wohnungsbauten von der Gemeinde Wien gebaut. Diese Wohnungen entsprechen wesentlich besseren Standards als davor in Wien üblich. Es wird begonnen Richtlinien und Bestimmungen in Form von Bauordnungen zu verordnen.

Nach dem zweiten Weltkrieg entstanden durch den steigenden Mobilisierungsgrad ermöglicht viele Entwicklungen an den Rändern der Stadt Wien. Es wurde Auto-freundlich gebaut und umgebaut. Allerdings wurde diese Entwicklung auch kritisch gesehen. So wird der Fokus der Stadtentwicklung in Wien auf die Erhaltung und Erneuerung gelegt.

Nach dem Fall der eisernen Mauer rückt Wien weiter ins Zentrum Europas. Es kommt immer wieder Überlegungen über eine Kooperation mit Bratislava.

\begin{question}
	Worin begründen sich die besonderen Herausforderungen der Wiener Stadtentwicklung?
\end{question}
Die Herausforderung besteht darin, Strategien und Instrumente der Stadtentwicklung so weiterzuentwickeln, dass sie erreichte Qualitätsstandards nicht nur erhalten helfen, sondern neue, zukunftsgerichtete Qualitäten ermöglichen;

\begin{itemize}
	\item standortwirtschaftliche und infrastrukturelle Rahmenbedingungen für lokale wie internationale Investorinnen und Investoren sowie Entwicklerinnen und Entwickler derart zu gestalten, dass rasch, elastisch und innovativ auf Veränderungen reagiert werden kann und dabei den Interessen und Bedürfnissen der Bevölkerung entsprochen wird;
	\item die Stadt systematisch in ihrer gebauten, (frei-)räumlichen und ökologischen Substanz fit zu machen, um ein qualitätsvolles Wachstum zu ermöglichen, das Wertvolles erhält, Verbrachtes erneuert und überholtes transformiert;
	\item ein stabiles soziales Gleichgewicht in der Stadt zu erzielen und Diversität und Gleichstellung von Frauen und Männern als wesentliche Prinzipien bei der Nutzung und Entwicklung der Stadt zu vertiefen;
	\item die Entwicklung der Stadt als kollektive Verantwortung und Kooperationsaufgabe von Politik, Wirtschaft und Bevölkerung in den Vordergrund zu rücken und dementsprechend Prozesse der Planung, des Managements und der Umsetzung von Stadtentwicklung partizipativ und effizient zu gestalten.
\end{itemize}

\begin{question}
	Erläutern Sie wesentliche im STEP 2025 dokumentierte Prinzipien der Wiener Stadtentwicklung!
\end{question}
Bei der Bearbeitung der vielfältigen Agenden steht ein zentraler Anspruch im Mittelpunkt: Wachstum und Entwicklungsdynamik sollen zu einem Aufbruch führen, der der ganzen Stadt zugute kommt. Wien soll auch in Zukunft eine lebenswerte Stadt wein, in der Menschen gerne leben, arbeiten, lernen und sich austauschen. Die Qualität, die Wien attraktiv macht, soll für alle - Alte und Junge, ''Alteingesessene'' und ''Zugezogene'' sowie Besucherinnen und Besucher - erlebbar sein. Diese Zielsetzung bedeutet in der strategischen Ausrichtung, dass die unterschiedlichen Dimensionen einer nachhaltigen Entwicklung gleichgewichtig verfolgt werden: Wettbewerbsfähigkeit und Unternehmergeist ebenso wie Leistbarkeit, soziale Gerechtigkeit und Integration sowie eine ressourcenschonende Klima- und Umweltschutzpolitik.

Der STEP 2025 ist dabei kein für sich allein stehendes Dokument. Er berücksichtigt vielmehr die Besonderheiten, Stärken und Schwächen des Standorts Wien, bezieht raumrelevante Aussagen von Fachkonzepten verschiedener Ressorts mit ein und baut auf einer Reihe von Grundhaltungen auf, die die Perspektive einer ''Stadt der Zukunft'' umreißen und für die Wiener Stadtentwicklung handlungsleitend sind.

\begin{question}
	Erläutern Sie die wesentlichen Ziele der vier Handlungsbereiche des STEP 2025!
\end{question}
\begin{itemize}
	\item Wir leisten uns Stadt
	
	TODO
	
	\item Wien baut auf
	
	Qualitätsvolle Stadtstruktur und vielfältige Urbanität
	
	\item Wien wächst über sich hinaus
	
	Wachstum und Wissensgesellschaft transformieren die Metropolregion
	
	\item Wien ist vernetzt
	
	Weitsichtig, robust und tragfähig für Generationen
\end{itemize}

\begin{question}
	Diskutieren Sie den im STEP 2025 zum Ausdruck gebrachten Anspruch ''Wir leisten uns Stadt!'' (S. 12ff)
\end{question}
TODO

\begin{question}
	Diskutieren Sie den Anspruch des STEP 2025 an eine qualitätsvolle Stadtstruktur und vielfältige Urbanität! Benennen Sie die zentralen Ziele! (S. 34)
\end{question}
TODO

\begin{question}
	Was sind die im STEP 2025 dokumentierten Aspekte der Flächensicherung für das Stadtwachstum? (S. 48ff)
\end{question}
TODO

\begin{question}
	Was versteht der STEP 2025 unter einer ''ausgewogenen, polyzertrischen Standortentwicklung''? (S. 64)
\end{question}
TODO

\begin{question}
	Diskutieren Sie das Leitbild zur Siedlungsentwicklung! (S. 67)
\end{question}
TODO

\begin{question}
	Diskutieren Sie den Anspruch and die Entwicklung der Metropolregion! (S. 88ff)
\end{question}
TODO

\begin{question}
	Was sind Fachkonzepte zum STEP25?
\end{question}
TODO

\begin{question}
	Welche Wirkungen entfalten STEP 2025 und die Fachkonzepte zum STEP?
\end{question}
TODO

\begin{question}
	Welchen Stellenwert besitzt die Smart City Rahmenstrategie der Stadt Wien?
\end{question}
TODO

\begin{question}
	Erläutern Sie die drei Handlungsfelder auf die sich die Smart City Rahmenstrategie der Stadt Wien bezieht!
\end{question}
TODO

\section{Theorie und Methodik der Stadtplanung}

\section{Instrumente der Örtlichen Raumplanung}

\end{document}
