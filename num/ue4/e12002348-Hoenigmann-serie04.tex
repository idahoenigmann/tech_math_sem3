\documentclass[]{article}

\usepackage{amsfonts} 
\usepackage{amsmath}
\usepackage[margin=3cm]{geometry}

%opening
\title{NUM UE4}
\author{Ida Hönigmann}

\begin{document}

\maketitle

\section{Aufgabe 14:}
Wir zeigen $V_nA = DV_n$ indem wir auf jeder Seite den Wert in der j-ten Zeile und k-ten Spalte berechnen.

Wir wissen, dass

\[
(V_n)_{jk} = w_n^{j*k}.
\]

Durch die Diagonalstruktur der Matrix $D$ wird bei einer Multiplikation mit $D$ von links jede Zeile mit dem entsprechenden Wert aus $D$ skaliert, d.h.

\[
(DV_n)_{jk} = p(w_n^j)*w_n^{j*k} = (\sum_{l=0}^{n-1}a_l*w_n^{j*l}) * w_n^{j*k} = \sum_{l=0}^{n-1}a_l*w_n^{j(l+k)}.
\]

Um $(V_nA)_{jk}$ auszurechnen wollen wir uns zunächst die entsprechende Zeile von $V_n$ und Spalte von $A$ ansehen:

\begin{align*}
(V_nA)_{jk} &= (w_n^{j*0}, w_n^{j*1}, ..., w_n^{j*(n-1)}) * (a_{n-k}, a_{n-k+1}, ..., a_{n-1}, a_0, a_1, ..., a_{n-k-1})^T \\
 &= \sum_{l=0}^{k-1} (w_n^{j*l} * a_{n-k+l}) + \sum_{l=0}^{n-k-1} (w_n^{j*(l+k)}*a_l) \\
 &= \sum_{l=n-k}^{k-1+n-k} (w_n^{j*(l-n+k)} * a_{n-k+l-n+k}) + \sum_{l=0}^{n-k-1} (w_n^{j*(l+k)}*a_l) \\
  &= \sum_{l=n-k}^{n-1} (w_n^{j*(l+k-n)} * a_{l}) + \sum_{l=0}^{n-k-1} (w_n^{j*(l+k)}*a_l)
\end{align*}

Da $w_n$ die n-te Einheitswurzel ist gilt $(w_n)^n = 1$ und somit

\[
w_n^{j(l+k-n)} = \frac{w_n^{j(l+k)}}{w_n^{j*n}} = \frac{w_n^{j(l+k)}}{((w_n)^{n})^{j}} = \frac{w_n^{j(l+k)}}{1} = w_n^{j(l+k)}.
\]

Insgesamt ergibt das

\[
(V_nA)_{jk} = \sum_{l=n-k}^{n-1} (w_n^{j*(l+k)} * a_{l}) + \sum_{l=0}^{n-k-1} (w_n^{j*(l+k)}*a_l) = \sum_{l=0}^{n-1} (w_n^{j*(l+k)} * a_{l}) = (DV_n)_{jk}.
\]

Somit ist gezeigt, dass $V_nAV_n^{-1} = D$.

\section{Aufgabe 15:}
Nach Aufgabe 14 gilt $V_nAV_n^{-1} = D$, was uns $A = V_n^{-1}DV_n$ und weiter $A^{-1}=V_n^{-1}D^{-1}V_n$ liefert. Die Inverse von $A$ existiert wegen der Regularität.

Von $V_n$ wissen wir, dass $V_n^{-1} = \frac{1}{n} \overline{V_n}$ und durch die Diagonalform von $D$ gilt $D^{-1}=\overline{D}$.

Um $Ax=b$ zu lösen können wir einfach $x = A^{-1}b = \frac{1}{n} \overline{V_n} \bar{D}V_nb$ berechnen.

Schauen wir uns zuerst die Berechnung von $D$ an.

Es sei $a$ der Vektor aus dem die zirkulante Matrix $A$ aufgebaut ist und $v_1$ die erste Spalte von $V_n$. Für $v_1$ gilt $v_1 = (1)_{j=0}^{n-1}$. Es gilt $V_nA = DV_n$ und somit auch für die erste Spalte $V_na = Dv_1 = D*(1, 1, ..., 1)^T = (p(1), p(w_n), ..., p(w_n^{n-1}))$.

Somit berechnet folgender Algorithmus die Lösung von $Ax=b$:

\begin{enumerate}
	\item Berechnen von $d:=V_na$ mittels FFT
	\item Berechnen von $\bar{d}$ komponentenweise
	\item Berechnen von $x:=V_nb$ mittels FFT
	\item Berechnen von $y:=\bar{d}*x$ komponentenweise
	\item Berechnen von $\bar{y}$ komponentenweise
	\item Berechnen von $z:=V_n\bar{y}$ mittels FFT
	\item Berechnen von $\bar{z}$ komponentenweise
	\item Berechnen von $\frac{1}{n}\bar{z}$ komponentenweise
\end{enumerate}

Der Aufwand beträgt 3 FFTs und 5 komponentenweisen Berechnungen, also $3*(n log_2 n) + 5n$ bzw. $O(n log n)$.

\section{Aufgabe 16:}
Wenn $A \in \mathbb{K}^{n\times n}$ eine Toeplitz-Matrix ist, können wir sie wie folgt in eine zirkulante Matrix $B \in \mathbb{K}^{2n\times 2n}$ einbetten:

\[
B := 
\begin{pmatrix}
	\begin{matrix}
		a_{-n+1} & a_{n-1}  & \cdots & a_{1}   \\
		a_{-n+2} & a_{-n+1} & \cdots & a_{2}   \\
		\vdots   & \vdots   & \ddots & \vdots  \\
		a_{-1}   & a_{-2}   & \cdots & a_{n-1} 
	\end{matrix} &
	\begin{matrix}
		a_{0}   & \cdots & a_{-n+2} \\
		a_{1}   & \cdots & a_{-n+3} \\
		\vdots  & \ddots & \vdots   \\
		a_{n-2} & \cdots & a_{0} 
	\end{matrix} \\

	\underbrace{
		\begin{bmatrix}
			a_{0}   & a_{-1}  & \cdots & a_{-n+1}  \\
			a_{1}   & a_{0}   & \cdots & a_{-n+2}  \\
			a_{2}   & a_{1}   & \cdots & a_{-n+3}  \\
			\vdots  & \vdots  & \ddots & \vdots    \\
			a_{n-1} & a_{n-2} & \cdots & a_{0} 
		\end{bmatrix}
	}_{=A}
	 &
	\begin{matrix}
		a_{n-1}  & \cdots & a_{1}    \\
		a_{-n+1} & \cdots & a_{2}    \\
		a_{-n+2} & \cdots & a_{3}    \\
		\vdots   & \ddots & \vdots   \\
		a_{-1}   & \cdots & a_{-n+1} 
	\end{matrix}
\end{pmatrix}
\]

Wenn wir $y:=(x_1, x_2, ..., x_n, 0, .., 0)^T \in \mathbb{K}^{2n}$ (als ''Erweiterung'' von x) setzten. Erhalten wir $z:=A*x = (z_1, z_2, ..., z_n)^T \in \mathbb{K}^n$ indem wir uns die letzen n Werte von $B*y=(?, ?, ..., ?, z_1, z_2, ..., z_n)^T$ ansehen.

Nun können wir mit der zirkulanten Matrix $B$ arbeiten, von der wir aus Beispiel 14 wissen, dass $B=V_n^{-1}DV_n=\frac{1}{n}\overline{V_n}DV_n$ gilt.

Sei $b$ der Vektor aus dem die Matrix $B$ aufgebaut ist.

Um nun $B*y=\frac{1}{n}\overline{V_n}DV_ny$ und somit $A*x$ zu berechnen können wir folgendem Algorithmus folgen:

\begin{enumerate}
	\item Berechnen von $d:=V_nb$ mittels FFT
	\item Berechnen von $f:=V_ny$ mittels FFT
	\item Berechnen von $g:=d*f$ komponentenweise
	\item Berechnen von $\bar{g}$ komponentenweise
	\item Berechnen von $h:= V_n\bar{g}$ mittels FFT
	\item Berechnen von $\bar{h}$ komponentenweise
	\item Berechnen von $z:= \frac{1}{n} \bar{h}$ komponentenweise
\end{enumerate}

Dann sind die letzten n Werte von $z$ das Ergebnis von $Ax$. Der Aufwand des Algorithmus beträgt 3 Mal den Aufwand einer FFT ($2n * log_2 (2n)$) und 4 komponentenweise Rechnungen (Aufwand von $2n$), also $O(n log n)$.

\end{document}
