\documentclass[]{article}

\usepackage{amsfonts} 
\usepackage{amsmath}
\usepackage[margin=3cm]{geometry}
\usepackage{enumitem}
\usepackage{amsthm}

\newcommand{\norm}[1]{\left|\left|#1\right|\right|}
\newcommand{\norminf}[1]{\norm{#1}_\infty}
\newcommand{\normone}[1]{\norm{#1}_1}
\newcommand{\supinf}{\sup_{\norminf{x}=1}}
\newcommand{\supone}{\sup_{\normone{x}=1}}
\newcommand{\maxz}[2]{\max_{#1=1,...,#2}}

%opening
\title{NUM UE9}
\author{Ida Hönigmann}

\begin{document}

\maketitle

\section{Aufgabe 33:}

\begin{proof}
	
	\begin{enumerate}[label=\alph*)]
		\item Wie letzte Woche gezeigt gilt $\norminf{A}=\supinf\norminf{Ax}$.
		
		Sei $x \in \mathbb{K}^n$ mit $\norminf{x}=1$ beliebig. Also gilt $\maxz{j}{n}|x_j|=1$.
		
		\begin{align*}
			\norminf{Ax}=\maxz{j}{m}|(Ax)_j|=\maxz{j}{m}\left|\sum_{k=1}^{n}a_{jk}x_k\right|
			\leq \maxz{j}{m}\sum_{k=1}^{n}|a_{jk}|\cdot|x_k| \leq \maxz{j}{m} \sum_{k=1}^{n}|a_{jk}|
		\end{align*}
		
		Sei $j \in \{1, ..., m\}$ mit $\sum_{k=1}^{n}|a_{jk}|$ maximal. Falls $\sum_{k=1}^{n}|a_{jk}|=0$ folgt $\forall k=1,...,n \forall j=1,...,m: a_{jk}=0$ und somit ist die Aussage klar. Sonst gilt für
		
		\begin{align*}
			x=\begin{pmatrix}
				sgn(a_{j1})\\
				sgn(a_{j2})\\
				\cdots\\
				sgn(a_{jn})\\
			\end{pmatrix}
			\in \mathbb{K}^n &&
			\norminf{x}=1
		\end{align*}
		
		\begin{align*}
			\norminf{Ax}=\maxz{l}{m}\left|\sum_{k=1}^{n}a_{lk}x_k\right|=\maxz{l}{m}\left|\sum_{k=1}^{n}a_{lk}\cdot sgn(a_{jk})\right| = \left|\sum_{k=1}^{n}|a_{jk}|\right| = \sum_{k=1}^{n}|a_{jk}| = \maxz{j}{m}\sum_{k=1}^{n}|a_{jk}|
		\end{align*}
		
		Also folgt $\norminf{A}=\maxz{j}{m}\sum_{k=1}^{n}|a_{jk}|$.
		
		\item 
		Um $\normone{A}=\maxz{k}{n}\sum_{j=1}^{m}|a_{jk}|$ zu zeigen schauen wir uns zunächst folgendes an:
		
		Sei $x\in\mathbb{K}^n$ mit $\normone{x}=1$ beliebig. Also gilt $\sum_{j=1}^{n}|x_j|=1$.
		
		\begin{align*}
			|(Ax)_j| &= \left|\sum_{k=1}^{n}a_{jk}x_k\right| \leq \left|\sum_{k=1}^{n}\maxz{l}{n}|a_{jl}|x_k\right| = \maxz{l}{n}|a_{jl}|\left|\sum_{k=1}^{n}x_k\right| \\  &\leq\maxz{l}{n}|a_{jl}|\sum_{k=1}^{n}|x_k| = \maxz{l}{n}|a_{jl}| \cdot \normone{x}
			= \maxz{l}{n}|a_{jl}| \\
			\implies \normone{Ax} &= \sum_{j=1}^{m}|(Ax)_j| \leq \sum_{j=1}^{m}\maxz{k}{n}|a_{jk}| = \maxz{k}{n}\sum_{j=1}^{m}|a_{jk}|
		\end{align*}
		
		Wähle $j\in\{1, ..., n\}$ so, dass $\sum_{k=1}^{m}|a_{kj}|$ maximal ist. Sei $x\in\mathbb{K}^n$ mit $x_j=1$ und $\forall l\neq j: x_l=0$.
		
		Dann gilt
		
		\begin{align*}
			|(Ax)_l| = \left|\sum_{k=1}^{n}a_{lk}x_k\right| = |a_{lj}| \\
			\normone{Ax} = \sum_{k=1}^{m}|(Ax)_k| = \sum_{k=1}^{m}|a_{kj}| = \maxz{j}{n} \sum_{k=1}^{m}|a_{kj}|
		\end{align*}
		
		Also folgt $\normone{A} = \maxz{k}{n}\sum_{j=1}^{m}|a_{jk}|$.
	\end{enumerate}
	
\end{proof}

\newpage

\section{Aufgabe 34:}

\begin{proof}
	Sei $A\in\mathbb{K}^{n\times n}$ ... irreduzibel und diagonaldominant beliebig.
	
	zz: $A$ ist regulär
	
	Angenommen $A$ wäre nicht regulär, also $\exists x\in\mathbb{K}^n\setminus\{0\}: Ax=0$. Dann folgt
	
	\begin{align*}
		0 = (Ax)_j = \sum_{l=1}^{n}a_{jl}x_l\\
		\implies -a_{jj}x_j = \sum_{l=1,l\neq j}^{n}a_{jl} x_l\\
		\implies |a_{jj}|\cdot |x_j| = |a_{jj}x_j| = \left|\sum_{l=1,l\neq j}^{n}a_{jl} x_l\right| \leq \sum_{l=1,l\neq j}^{n}|a_{jl}| \cdot |x_l|.
	\end{align*}

	Definieren wir nun $J:=\{j\in\{1,...,n\}: |x_j| = \norminf{x}\}$ und $K:=\{k\in\{1,...,n\}: |x_k| < \norminf{x}\}$. Offensichtlich gilt $J \cup K = \{1, ..., n\}$ und $J \cap K = \emptyset$.

	Fallunterscheidung:
	
	\begin{enumerate}[label=\arabic*. Fall: ]
		\item $K = \emptyset$
		
		Also gilt $|x_1| = |x_2| = ... = |x_n|$.
		
		\begin{align*}
			|a_{jj}|\cdot |x_j| \leq \sum_{l=1,l\neq j}^{n}|a_{jl}| \cdot |x_l| \\
			\implies |a_{jj}| \leq \sum_{l=1,l\neq j}^{n}|a_{jl}|
		\end{align*}
	
		Was ein Widerspruch zu $\forall j \in \{1,...,n\}: |a_{jj}| \geq \sum_{l=1, l\neq j}^{n}|a_{jl}| \land \exists j\in\{1,...,n\}: |a_{jj}| > \sum_{l=1, l\neq j}^{n}|a_{jl}|$ ist.
		
		\item $K \neq \emptyset$
		
		Da $A$ irreduzibel ist $\exists k\in K \exists j \in J: a_{jk} \neq 0$
		
		\begin{align*}
			|a_{jj}| \leq \sum_{l=1,l\neq j}^{n}|a_{jl}|\frac{|x_l|}{|x_j|} = \sum_{l=1,l\neq j}^{n}|a_{jl}|\underbrace{\frac{|x_l|}{\norminf{x}}}_{\leq 1 \land \exists l: < 1} < \sum_{l=1,l\neq j}^{n}|a_{jl}|
		\end{align*}
	
		Was ein Widerspruch zu $\forall j \in \{1,...,n\}: |a_{jj}| \geq \sum_{l=1, l\neq j}^{n}|a_{jl}|$ ist.
	\end{enumerate}

	In beiden Fällen folgt aus dem Widerspruch, dass $A$ regulär ist.
	
	zz: $\forall j=1, ..., n: a_{jj}\neq 0$
	
	Angenommen $\exists j\in \{1, ..., n\}: a_{jj} = 0$. Da $A$ diagonaldominant ist folgt
	
	\begin{align*}
		\sum_{k=1,k\neq j}^{n}|a_{jk}| \leq |a_{jj}| = 0\\
		\implies \forall k\in\{1, ..., n\}: |a_{jk}| = 0
	\end{align*}

	Was im Widerspruch zur Regularität von $A$ steht.
	
	Also folgt, dass $A$ regulär und $\forall j\in\{1,...,n\}:a_{jj}\neq 0$.
\end{proof}

\end{document}
