\documentclass[]{article}

\usepackage{amsfonts} 
\usepackage{amsmath}
\usepackage[margin=3cm]{geometry}
\usepackage{enumitem}
\usepackage{amsthm}

%opening
\title{NUM UE8}
\author{Ida Hönigmann}

\begin{document}

\maketitle

\section{Aufgabe 29:}

\begin{proof}
	\begin{enumerate}[label=\alph*)]
		\item 
		
		Sei $x_0 \in [a,b]$ eine Nullstelle von $f$.
		
		\begin{align*}
			X := [a,b]^2 && x^* := (x_0, x_0) && \Phi &: X \rightarrow X \\
			&& && (a_k, b_k) &\mapsto \begin{cases}
				(a_k, c_k), &\quad\text{falls} f(a_k)f(c_k)\le 0 \\
				(c_k, b_k), &\quad\text{sonst}
			\end{cases}
		\end{align*}
		
		wobei $c_k:=\frac{a_k+b_k}{2}$. Dann ist $(X, \Phi, x^*)$ ein abstraktes Iterationsverfahren, dass das Bisektionsverfahren abbildet.
		
		\item
		
		\begin{align*}
			d &:X^2\rightarrow\mathbb{R} \\
			((a,b),(x,y))&\mapsto ||a - b| - |x - y||
		\end{align*}
		
		\begin{align*}
			\forall (a,b), (x,y) \in X: d((a,b),(x,y))\geq 0\\
			\forall (a,b) \in X: d((a,b),(a,b)) = ||a - b| - |a - b|| = 0\\
			\forall (a,b), (x,y) \in X: d((a,b),(x,y)) = ||a - b| - |x - y|| = ||x - y| - |a - b|| = d((x,y),(a,b))\\
			\forall (a,b), (x,y), (c,d) \in X: d((a,b),(x,y)) + d((x,y), (c,d)) = ||a - b| - |x - y|| + ||x - y| - |c - d|| \\
			\geq ||a - b| - |x - y| + |x - y| - |c - d|| = ||a - b| - |c - d|| = d((a,b), (c,d))
		\end{align*}
		
		Somit ist $d$ eine Metrik auf $X$.
		
		Sei $(a,b), (x,y) \in X$ beliebig.
		
		\begin{align*}
			d(\Phi(a,b), \Phi(x,y))=
			\begin{cases}
				d_2((a, \frac{a+b}{2}), (x,\frac{x+y}{2})) =
				\left|\left|a - \frac{a+b}{2}\right| - \left|x - \frac{x+y}{2}\right|\right| =
				\left|\left|\frac{a+b}{2} - a\right| - \left|\frac{x+y}{2} - x\right|\right| \\
				d_2((a, \frac{a+b}{2}), (\frac{x+y}{2},y))  =
				\left|\left|a - \frac{a+b}{2}\right| - \left|\frac{x+y}{2} - y\right|\right| =
				\left|\left|\frac{a+b}{2} - a\right| - \left|\frac{x+y}{2} - y\right|\right| \\
				d_2((\frac{a+b}{2}, b), (x,\frac{x+y}{2}))  =
				\left|\left|\frac{a+b}{2} - b\right| - \left|x - \frac{x+y}{2}\right|\right| =
				\left|\left|\frac{a+b}{2} - b\right| - \left|\frac{x+y}{2} - x\right|\right| \\
				d_2((\frac{a+b}{2}, b), (\frac{x+y}{2},y))  =
				\left|\left|\frac{a+b}{2} - b\right| - \left|\frac{x+y}{2} - y\right|\right| =
				\left|\left|\frac{a+b}{2} - b\right| - \left|\frac{x+y}{2} - y\right|\right|
			\end{cases}\\
			= \left|\left|\frac{a-b}{2}\right|-\left|\frac{x-y}{2}\right|\right|=\frac{1}{2}||a-b|-|x-y||
		\end{align*}
		
		\begin{align*}
			d((a,b),(x,y)) = ||a-b|-|x-y||
		\end{align*}
		
		Also $\forall (a,b), (x,y) \in X: d(\Phi(a,b), \Phi(x,y)) \le q d_2((a,b),(x,y))$ mit $q:=\frac{1}{2}$. Somit gilt nach Banach'schem Fixpunktsatz, dass $(X, \Phi, x^*)$ für alle Startwerte aus $X$ global und linear mit $q=\frac{1}{2}$ gegen $x^*$ konvergiert.
		
		\vspace{3cm}
		
		Noch einmal nachgerechnet:
		
		\begin{itemize}
			\item Globale Konvergenz: Wir wollen zeigen, dass $(a_k, b_k)_{k \in \mathbb{N}}$ eine Cauchy-Folge ist und somit konvergiert.	
			Da wie oben gezeigt gilt, dass $\forall k \in \mathbb{N}: |a_{k+1} - b_{k+1}| = \frac{1}{2}|a_k - b_k|$ folgt durch vollständige Induktion, dass
			
			\begin{align*}
				\forall k \in \mathbb{N}: |a_k - b_k| = \frac{1}{2^{1}}|a_{k-1} - b_{k-1}| = \frac{1}{2^{2}}|a_{k-2} - b_{k-2}| = ... = \frac{1}{2^{k}}|a_0 - b_0|.
			\end{align*}
			
			Sei $\epsilon > 0$ beliebig. Sei $k, l \in \mathbb{N}$ beliebig.
			
			\begin{align*}
				d((a_k, b_k), (a_l, b_l)) = ||a_k - b_k| - |a_l - b_l|| = \left|\frac{1}{2^{k}}|a_0 - b_0| - \frac{1}{2^{l}}|a_0 - b_0|\right| = \left|\frac{1}{2^{k}}-\frac{1}{2^{l}}\right||a_0 - b_0| < \epsilon
			\end{align*}
			
			für groß genug gewählte $k$ und $l$. Also handelt es sich um eine Cauchy-Folge und sie ist somit konvergent gegen $x^*$ (laut VO gilt $\exists \lim\limits_{k\rightarrow\infty}x_k$, so gilt $\lim\limits_{x\rightarrow\infty}x_k = x^*$ da es sich um den einzigen Fixpunkt handelt).
			
			\item Lineare Konvergenz: Sei $\epsilon >0$, $(a_0, b_0) \in U_\epsilon(x^*)$ und $k \in \mathbb{N}$ beliebig.
			
			\begin{align*}
				d((a_{k+1}, b_{k+1}), (x_0, x_0)) = ||a_{k+1} - b_{k+1}| - |x_0 - x_0|| = \frac{1}{2}|a_{k} - b_{k}| = \frac{1}{2} d((a_k, b_k), (x_0, x_0))
			\end{align*}
			
			Also ist $(X, \Phi, x^*)$ linear mit $q=\frac{1}{2}$ konvergent.
		\end{itemize}
		
	\end{enumerate}
	
\end{proof}

\newpage

\section{Aufgabe 32:}

\end{document}
