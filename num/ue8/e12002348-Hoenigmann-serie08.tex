\documentclass[]{article}

\usepackage{amsfonts} 
\usepackage{amsmath}
\usepackage[margin=3cm]{geometry}
\usepackage{enumitem}

%opening
\title{NUM UE8}
\author{Ida Hönigmann}

\begin{document}

\maketitle

\section{Aufgabe 29:}

\begin{enumerate}[label=\alph*)]
	\item 
	
	Sei $x_0 \in [a,b]$ eine Nullstelle von $f$.
	
	\begin{align*}
		X := [a,b]^2 && x^* := (x_0, x_0) && \Phi &: X \rightarrow X \\
		&& && (a_k, b_k) &\mapsto \begin{cases}
			(a_k, c_k), &\quad\text{falls} f(a_k)f(c_k)\le 0 \\
			(c_k, b_k), &\quad\text{sonst}
		\end{cases}
	\end{align*}

	wobei $c_k:=\frac{a_k+b_k}{2}$. Dann ist $(X, \Phi, x^*)$ ein abstraktes Iterationsverfahren, dass das Bisektionsverfahren abbildet.
	
	\item
	
	\begin{align*}
		d_1&:\mathbb{R}^2\rightarrow\mathbb{R} & d_2 &:(\mathbb{R}^2)^2\rightarrow\mathbb{R} \\
		(a,b)&\mapsto |a-b| & ((a,b),(x,y))&\mapsto|d_1(a,b) - d_1(x,y)| 
	\end{align*}

	sind beides Metriken auf $\mathbb{R}$ bzw. $\mathbb{R}^2$.
	
	Sei $(a,b), (x,y) \in X^2$ beliebig.
	
	\begin{align*}
		d_2(\Phi(a,b), \Phi(x,y))=
		\begin{cases}
			d_2((a, \frac{a+b}{2}), (x,\frac{x+y}{2})) \\
			d_2((a, \frac{a+b}{2}), (\frac{x+y}{2},y)) \\
			d_2((\frac{a+b}{2}, b), (x,\frac{x+y}{2})) \\
			d_2((\frac{a+b}{2}, b), (\frac{x+y}{2},y)) 
		\end{cases}
		= \left|\frac{a+b}{2}-\frac{x+y}{2}\right| = \frac{1}{2} |a+b-(x+y)| \\
		d_2((a,b),(x,y))=||a-b|-|x-y||=||b-a|-|x-y||\le|b-a+x-y|=...=|a+b-(x+y)|
	\end{align*}
	
	Also $\forall (a,b), (x,y) \in X: d_2(\Phi(a,b), \Phi(x,y)) \le q d_2((a,b),(x,y))$ mit $q:=\frac{1}{2}$. Somit gilt nach Banach'schem Fixpunktsatz, dass $(X, \Phi, x^*)$ für alle Startwerte aus $X$ global und linear mit $q=\frac{1}{2}$ gegen $x^*$ konvergiert.
	
	Globale Konvergenz: Wir wollen zeigen, dass $(a_k, b_k)_{k \in \mathbb{N}}$ eine Cauchy-Folge ist und somit konvergiert. Sei $\epsilon > 0$ beliebig. Sei $k, l \in \mathbb{N}$ beliebig.
	
	\begin{align*}
		d_2((a_k, b_k), (a_l, b_l)) = |d_1(a_k, b_k) - d_1(a_l, b_l)|
	\end{align*}
	
	O.b.d.A gilt $k < l$. Da $\forall k \in \mathbb{N}: d_1(a_{k+1}, b_{k+1}) = \frac{1}{2}d_1(a_k, b_k)$ folgt durch vollständige Induktion, dass
	
	\begin{align*}
		\forall k \in \mathbb{N}: d_1(a_k, b_k) = \frac{1}{2^{k}}d_1(a_0, b_0) =\frac{1}{2^{k}}|a_0 - b_0|.
	\end{align*}
	
	Somit gilt
	
	\begin{align*}
		|d_1(a_k, b_k) - d_1(a_l, b_l)| = \left|\frac{1}{2^{k}}|a_0 - b_0| - \frac{1}{2^{l}}|a_0 - b_0|\right| = \left|\frac{1}{2^{k}}-\frac{1}{2^{l}}\right||a_0 - b_0| = \left|\frac{1}{2^{k}}-\frac{1}{2^{l}}\right||a_0 - b_0| < \epsilon
	\end{align*}
	
	für groß genug gewählte $k$ und $l$. Also handelt es sich um eine Cauchy-Folge und ist somit konvergent gegen $x^*$.  
	
	
	
	Lineare Konvergenz: Sei $\epsilon >0$, $(a_0, b_0) \in U_\epsilon(x^*)$ und $k \in \mathbb{N}$ beliebig.
	
	\begin{align*}
		d_2(x_{k+1},x^*) = d_2((a_{k+1}, b_{k+1}), (x_0, x_0)) = d_2(d_1(a_{k+1}, b_{k+1}) - \underbrace{d_1(x_0, x_0)}_{=0}) = |a_{k+1} - b_{k+1}| \\
		d_2(x_{k}, x^*) = |a_{k} - b_{k}| = 2 |a_{k+1} - b_{k+1}| \\
		\implies d_2(x_{k+1}, x^*) = \frac{1}{2} d_2(x_k, x^*)
	\end{align*}

	Also ist $(X, \Phi, x^*)$ linear konvergent.

	
	
\end{enumerate}



\end{document}
